%File: formatting-instructions-latex-2023.tex
%release 2023.0
\documentclass[letterpaper]{article} % DO NOT CHANGE THIS
\usepackage{aaai23}  % DO NOT CHANGE THIS
\usepackage{times}  % DO NOT CHANGE THIS
\usepackage{helvet}  % DO NOT CHANGE THIS
\usepackage{courier}  % DO NOT CHANGE THIS
\usepackage[hyphens]{url}  % DO NOT CHANGE THIS
\usepackage{graphicx} % DO NOT CHANGE THIS
\urlstyle{rm} % DO NOT CHANGE THIS
\def\UrlFont{\rm}  % DO NOT CHANGE THIS
\usepackage{natbib}  % DO NOT CHANGE THIS AND DO NOT ADD ANY OPTIONS TO IT
\usepackage{caption} % DO NOT CHANGE THIS AND DO NOT ADD ANY OPTIONS TO IT
\frenchspacing  % DO NOT CHANGE THIS
\setlength{\pdfpagewidth}{8.5in}  % DO NOT CHANGE THIS
\setlength{\pdfpageheight}{11in}  % DO NOT CHANGE THIS
%
% These are recommended to typeset algorithms but not required. See the subsubsection on algorithms. Remove them if you don't have algorithms in your paper.
\usepackage{algorithm}
\usepackage{algorithmic}
\usepackage{csquotes}

%
% These are are recommended to typeset listings but not required. See the subsubsection on listing. Remove this block if you don't have listings in your paper.
\usepackage{newfloat}
\usepackage{listings}
\DeclareCaptionStyle{ruled}{labelfont=normalfont,labelsep=colon,strut=off} % DO NOT CHANGE THIS
\lstset{%
	basicstyle={\footnotesize\ttfamily},% footnotesize acceptable for monospace
	numbers=left,numberstyle=\footnotesize,xleftmargin=2em,% show line numbers, remove this entire line if you don't want the numbers.
	aboveskip=0pt,belowskip=0pt,%
	showstringspaces=false,tabsize=2,breaklines=true}
\floatstyle{ruled}
\newfloat{listing}{tb}{lst}{}
\floatname{listing}{Listing}
%
% Keep the \pdfinfo as shown here. There's no need
% for you to add the /Title and /Author tags.
\pdfinfo{
/TemplateVersion (2023.1)
}

% DISALLOWED PACKAGES
% \usepackage{authblk} -- This package is specifically forbidden
% \usepackage{balance} -- This package is specifically forbidden
% \usepackage{color (if used in text)
% \usepackage{CJK} -- This package is specifically forbidden
% \usepackage{float} -- This package is specifically forbidden
% \usepackage{flushend} -- This package is specifically forbidden
% \usepackage{fontenc} -- This package is specifically forbidden
% \usepackage{fullpage} -- This package is specifically forbidden
% \usepackage{geometry} -- This package is specifically forbidden
% \usepackage{grffile} -- This package is specifically forbidden
% \usepackage{hyperref} -- This package is specifically forbidden
% \usepackage{navigator} -- This package is specifically forbidden
% (or any other package that embeds links such as navigator or hyperref)
% \indentfirst} -- This package is specifically forbidden
% \layout} -- This package is specifically forbidden
% \multicol} -- This package is specifically forbidden
% \nameref} -- This package is specifically forbidden
% \usepackage{savetrees} -- This package is specifically forbidden
% \usepackage{setspace} -- This package is specifically forbidden
% \usepackage{stfloats} -- This package is specifically forbidden
% \usepackage{tabu} -- This package is specifically forbidden
% \usepackage{titlesec} -- This package is specifically forbidden
% \usepackage{tocbibind} -- This package is specifically forbidden
% \usepackage{ulem} -- This package is specifically forbidden
% \usepackage{wrapfig} -- This package is specifically forbidden
% DISALLOWED COMMANDS
% \nocopyright -- Your paper will not be published if you use this command
% \addtolength -- This command may not be used
% \balance -- This command may not be used
% \baselinestretch -- Your paper will not be published if you use this command
% \clearpage -- No page breaks of any kind may be used for the final version of your paper
% \columnsep -- This command may not be used
% \newpage -- No page breaks of any kind may be used for the final version of your paper
% \pagebreak -- No page breaks of any kind may be used for the final version of your paperr
% \pagestyle -- This command may not be used
% \tiny -- This is not an acceptable font size.
% \vspace{- -- No negative value may be used in proximity of a caption, figure, table, section, subsection, subsubsection, or reference
% \vskip{- -- No negative value may be used to alter spacing above or below a caption, figure, table, section, subsection, subsubsection, or reference

\setcounter{secnumdepth}{0} %May be changed to 1 or 2 if section numbers are desired.

% The file aaai23.sty is the style file for AAAI Press
% proceedings, working notes, and technical reports.
%

% Title

% Your title must be in mixed case, not sentence case.
% That means all verbs (including short verbs like be, is, using,and go),
% nouns, adverbs, adjectives should be capitalized, including both words in hyphenated terms, while
% articles, conjunctions, and prepositions are lower case unless they
% directly follow a colon or long dash
\title{Mosaic: Generating The New Yorker\\ Style Cartoons using Text-to-Image Diffusion Models}
\author{
    %Authors
    % All authors must be in the same font size and format.
    Mehul Sudrik,
    Rajesh Nagula,
    Utsav Oza
}
\affiliations{
    %Afiliations
    \textsuperscript{}ECE-GY 7123 Deep Learning\\
}

% REMOVE THIS: bibentry
% This is only needed to show inline citations in the guidelines document. You should not need it and can safely delete it.
\usepackage{bibentry}
% END REMOVE bibentry

\begin{document}

\maketitle

\section{Problem Statement}

\begin{displayquote}
    % “Make sure you can see how insignificant I am.”
    % “Let me interrupt your expertise with my confidence”
    % “Son, if you can't say something nice, say something clever but devastating.”
    ``Sometimes I wonder if I'm too old to be a cartoonist, but then I remember that I'm just not funny enough."
\end{displayquote}
This self-deprecating joke highlights the challenges of creating The New Yorker style cartoons, that features an intriacte blend of whimsical art style, witty humor, and a subtle commentary on modern life. Cartoon enthusiasts and creative professionals alike know that creating such cartoons can be a daunting task that requires both artistic talent and a knack for satire. In this project, we aim to explore techniques to simplify the cartoon creation process by using Text-to-Image Diffusion models to specifically generate high-quality The New Yorker style cartoons from natural language captions.

\section{Literature Survey}
Text-to-image diffusion models are a type of generative model that can produce high-quality images from textual descriptions. Diffusion \cite{c:21} is a probabilistic process that involves gradually adding noise to an image until it becomes completely random, and then gradually removing the noise until it becomes the target image. The goal of diffusion models is to learn the latent structure of a dataset by modeling the way in which data points diffuse through the latent space.

The challenge, however, is that it is unclear how the diffusion process can be directly exercised to generate images of specific unique concepts, or compose them in new roles and novel scenes. Textual Inversion \cite{c:22} allows us to teach text-to-image diffusion models new concepts - it takes a small number of images of a user-provided concept, like an object or a style, and learns to represent it through new ``words" in the embedding space of a frozen text-to-image model. These ``words" can be composed into natural language sentences, guiding personalized creation in an intuitive way. 
Another challenge that our project seeks to address is the preservation of the satirical and pun-laden nature of the captions while avoiding a literal interpretation of the text. This endeavor requires a delicate balance between generating an image that accurately represents the essence of the text and allowing for creative interpretation that enhances the humor and social commentary of the cartoon.


\section{Project Considerations}

\subsection{Dataset}
A key challenge in fine-tuning a text-to-image diffusion model is the requirement of a large and diverse dataset, which can be difficult to obtain for a specific domain, especially if it involves rare or complex visual concepts. Textual Inversion will allow us to teach the image generator a specific visual concept using relatively few image examples. As such, we'll be relying primarily on the dataset derived for The New Yorker's Cartoon Captioning Contest \cite{hessel2022androids}, which is composed of raw cartoon images that are mapped to quality-based ranked caption choices and an explanation describing the underlying humor of the cartoon.

\subsection{Model}

In this project, we will be utilizing Stability.ai's implementation of text-to-image diffusion model, Stable Diffusion. Specifically, we aim to explore the application of a fine-tuning technique call Textual Inversion on Stable Diffusion. This will involve creating a new text encoder and training it on our new embeddings, i.e. the captions for the cartoons we'd like to generate. The training algorithm will effectively take a sample from the output of the frozen Stable Diffusion image encoder's latent distribution for a training image, add noise to that sample, and then pass that noisy sample to the frozen diffusion model. Our final goal state will be when the model is able to separate the noise from the sample using our text encoding as hidden state.

% In addition to Textual Inversion, we aim to explore combinations of other fine-tuning techniques, such a DreamBooth, to further gain substantial control over the generated cartoon images and influence the inference process.

\section{Evaluation}

Considering the goal of this project is to explore techniques to simplify the cartoon creation process using Stable Diffusion, evaluating the generated outputs as such will be subjective in nature. We aim to use a combination of both qualitative (human assessment) and quantitative metrics (such as CLIP scores) to gauge the effectiveness of the model.

\bibliography{aaai23}

\end{document}
